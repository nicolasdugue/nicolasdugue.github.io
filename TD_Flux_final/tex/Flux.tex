\input{macros_td.tex}


\begin{document}
\feuille {1} 
%\vspace*{1cm}

\exos
Le but de cet exercice est d'�crire une application Java en ligne de commande qui permet
d'afficher les lignes d'un fichier contenant un certain mot. 
Par exemple, voici la liste des lignes qui
contiennent le mot "public" dans un fichier UnRectangle.java...\\
On lance la commande :\\
java Grep public UnRectangle.java\\
Et on obtient le r�sultat :\\
(UnRectangle.java,1) : public class UnRectangle {\\
(UnRectangle.java,4) :(public UnRectangle(UnPoint cig,int l,int h) {\\
(UnRectangle.java,10) : public void translation(int dx,int dy) {\\
Dans la parent�se (x:y), on trouve en premier x, le nom du fichier et en deuxi�me y, l'indice de la ligne.\\

La commande � utiliser a la forme g�n�rale :
java Grep mot fichier1 fichier2 ...\\

Remarque : lorsqu'un programme Java est lanc� avec des arguments en ligne de commande, ces arguments sont disponibles sous forme de cha\^ines de caract�res plac�es dans le tableau param�tre de la m�thode main. Dans l'exemple qui pr�c�de, si l'ent\^ete de la m�thode principale est {\tt public static void main(String[] args)}
alors lors de l'ex�cution, args.length vaut $3$, ce qui indique qu'il y a trois arguments en ligne de commande, {\tt args[0]} est la cha\^ine "mot", {\tt args[1]} est la cha\^ine "fichier1" et {\tt args[2]} est la cha\^ine "fichier2".\\

Cr�er une classe {\tt Grep} contenant un main capable d'effectuer les op�rations d�crites ci-dessus.

\exos
Ecrire une classe sort dont le main prend pour argument un nom de fichier texte et effectue une copie de ce fichier avec les lignes tri�es en utilisant l'ordre lexicographique.\\ 
Exemple :\\
Le fichier A est comme suit :
\begin{verbatim}
this is the end
oh, the end 
my only friend
\end{verbatim}
Ex�cuter la commande {\tt java Sort A} doit �crire un fichier Acopie comme suit : 
\begin{verbatim}
my only friend
oh, the end 
this is the end
\end{verbatim}
\exos
Parce que jamais deux sans trois... Mais on a d�j� boss� suffisamment et je n'ai plus d'id�es. \\\\
{\tt THE END}
\end{document}
