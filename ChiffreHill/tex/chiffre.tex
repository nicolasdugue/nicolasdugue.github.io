  \documentclass{article} 
 % categorie de document
  \usepackage[T1]{fontenc}
 \usepackage[latin1]{inputenc}
 % c pour les accents 
 \usepackage{url}
 \usepackage{xcolor}
 %\usepackage{a4wide}
% \usepackage{fullpage}
 %\usepackage{amsfonts}
 \usepackage{amsmath}
 %\usepackage{latexsym}
 \usepackage[frenchb]{babel}
 % c'est pour afficher le document comme en typographie FR %
  \usepackage{graphics}
 \usepackage[draft]{graphicx}
 % gère les images png, jpeg, gif
 % "draft" permet de gagner du temps pour les recompilation comportant des images
 % il ne les affiche pas mais grade les boites vides de la taille des images
 % l'option "final" est en général mise par défaut et sert �  ravoir les images
 
 \title{Chiffre de Hill}
 %\author{Linda}
 % on les a pas encore afficher
 \date{}
 % ça c parce que si on le dit pas, latex met la date de compilation par défaut

\newtheorem{definition}{Definition}

 \begin{document}
 
 \maketitle
 
 \abstract{
Le chiffre de Hill est une m\'ethode de cryptage sym\'etrique qui utilise certaines propri\'et\'es du calcul matriciel. Au fil de ce TD, nous proposons de mettre en place cette technique de cryptage/d\'ecryptage de messages textuels.
}

 \section{$Z_{26}$}
 
 Durant la suite du TD, on se place dans $ Z_{26}$ : nous consid\'erons uniquement des nombres de $0$ � $25$. Si lors des calculs certains r\'esultats sont n\'egatifs ou sup\'erieurs \`a $25$, les r\'esultats seront ramen\'es � cet intervalle via l'op\'eration de modulo $26$.\\
 Ainsi $13$ * $13$ = $169$ mais $169$ est sup\'erieur � $25$. On prend donc son r\'esultat modulo $26$ : $169$ $\%$ $26$ = $13$.\\
 On dit que $i$ est inversible dans $ Z_{26}$ s'il existe $j$ dans $ Z_{26}$ tel que $i * j (\% 26)= 1 $ dans $ Z_{26}$.
 
 \begin{table}[h]

	\centerline{
	\begin{tabular}{c|cccccccccccc}
			\hline\noalign{\smallskip}
			\noalign{\smallskip}\hline\noalign{\smallskip}
			x & $1$ & $3$ & $5$ & $7$ & $9$ & $11$ & $15$ & $17$ & $19$ & $21$ & $23$ & $25$ \\
			\noalign{\smallskip}\hline\noalign{\smallskip}
			x$^{-1}$ & $1$ & $9$ & $21$ & $15$ & $3$ & $19$ & $7$ & $23$ & $11$ & $5$ & $17$ & $25$\\
			\noalign{\smallskip}\hline\noalign{\smallskip}
		\end{tabular}
	}
	\caption{Inverses modulo $26$}\label{Inverses}
\end{table}

\section{Rappels de calcul matriciel}
 
 Durant tout le TD, pour des raisons de simplicit\'e de calcul, nous nous en tiendrons �  g\'erer des \textbf{matrices carr\'ees 2,2}.
 En effet, les formules simples qui suivent ne valent que pour des matrices 2,2. \\
Soit A une matrice 2,2 :
 $ A = \begin{pmatrix}
a & b \\
c & d
\end{pmatrix}$\\\\\\
Soit \textbf{$com(A)$} la comatrice de A tel que : 
 $ com(A) = \begin{pmatrix}
d & -c \\
-b & a
\end{pmatrix}$\\\\\\
Soit \textbf{$^{t}(A)$} la transpos\'ee de A tel que : 
 $ ^{t}(A) = \begin{pmatrix}
a & c \\
b & d
\end{pmatrix}$\\\\\\
Le d\'eterminant \textbf{$Det(A)$} de la matrice A est calcul\'e comme suit : 
 $ Det(A) = ad - cb$\\\\\\
 On dit qu'une matrice A est inversible dans $Z_{26}$ si $Det(A)$ poss�de un inverse dans $Z_{26}$ (voir Tableau~\ref{Inverses}).
 L'inverse de $Det(A)$ est not� $Det(A)^{-1}$.
La matrice inverse de $A$ not�e $(A)^{-1}$ est obtenue de la fa�on suivante : 
 $ (A)^{-1}=$ $^{t}(com(A)) \cdot Det(A)^{-1}$ \\\\\\


\section{Le chiffre de Hill}
\subsection{L'encodage du message}

Soit le message $M$ cod\'e num\'eriquement suivant la correspondance suivante : \\
\begin{table}[h]

	\centerline{
	\begin{tabular}{cccccccccccccccccccccccccc}
			\hline\noalign{\smallskip}
			\noalign{\smallskip}\hline\noalign{\smallskip}
			a & b & c & d & e & f & g & h & i & j & k & l & m & n & o & p & q & r & s & t & u & v & w & x & y & z\\
			\noalign{\smallskip}\hline\noalign{\smallskip}
			$0$ & $1$ & $2$ & $3$ & $4$ & $5$ & $6$ & $7$ & $8$ & $9$ & $10$ & $11$ & $12$ & $13$ & $14$ & $15$ & $16$ & $17$ & $18$ & $19$ & $20$ & $21$ & $22$ & $23$ & $24$ & $25$\\
			\noalign{\smallskip}\hline\noalign{\smallskip}
		\end{tabular}
	}
	\caption{Table de correspondance lettre/nombre}
\end{table}\\
Ainsi, par exemple $GEEK$ se code 6 4 4 10 en utilisant ce syst\`eme.\\
Le message $m$ est s\'epar\'e en bloc de $2$ lettres pour \^ etre encrypt\'e.
GEEK sera donc s�par� en deux paires qui seront encrypt�s tout � tour. D'abord la paire $\begin{pmatrix}
6\\
4 
\end{pmatrix}$ puis la paire $\begin{pmatrix}
4\\
10 
\end{pmatrix}$.

\subsection{Fonctionnement}
Soit une matrice inversible $A = \begin{pmatrix}
a & b \\
c & d
\end{pmatrix}$\\\\\\

Pour chiffrer chaque paire d'un message $M$ not�e $M_{i}$, il suffit d'appliquer une multiplication de matrice : 
$Chiffre(M_{i}) = A \cdot M_{i}$\\\\
Pour d\'echiffrer un message chiffr\'e avec la matrice A, on utilise la matrice inverse $A^{-1}$ : 
$M_{i} = A^{-1} \cdot Chiffre(M_{i})$

\subsection{Travail}
Pour r�aliser un programme capable d'encrypter/d\'ecrypter en utilisant le chiffre de Hill, nous proposons de r\'ealiser plusieurs modules dont voici les r\^oles.\\
Conversion chaines de caract�res - message num�rique :
\begin{itemize}
	\item Encoder un message re�u sous forme de chaine de caract�res en un message num�rique
	\item Encoder un message re�u sous forme num�rique en une chaine de caract�res
\end{itemize}
Calcul matriciel :
\begin{itemize}
	\item Calculer un d�terminant
	\item V�rifier l'inversibilit� d'une matrice
	\item Multiplier deux matrices
	\item Obtenir la comatrice d'une matrice
	\item Obtenir la transpos�e d'une matrice
	\item Inverser une matrice
\end{itemize}
Chiffre de Hill :
\begin{itemize}
	\item Chiffrer avec une matrice $A$ un message $M$
	\item D�chiffrer avec une matrice $A^{-1}$ un message $M$
\end{itemize}
Une interface utilisateur pour utiliser tous les modules ci-dessus.
\newpage

\section{Exemple}
Soit la matrice $A = \begin{pmatrix}
13 & 24 \\
10 & 13
\end{pmatrix}$ utilis� pour encrypter nos messages.\\\\
$Det(A) = 13 \cdot 13 - 10 \cdot 24= 7$.\\
Puisque 7 a pour inverse 15 modulo 26 : $ (7 \cdot 15) \% 26 = 1$, la matrice $A$ est inversible et peut donc nous servir � crypter nos messages.\\
On inverse donc $A$ : \\
$ (A)^{-1}=$ $^{t}(com(A)) \cdot Det(A)^{-1} = \begin{pmatrix}
13 & 2 \\
16 & 13
\end{pmatrix} \cdot 15$\\\\
$ (A)^{-1}= \begin{pmatrix}
13 & 4 \\
6 & 13
\end{pmatrix}$\\\\\\
Soit le message $"BC"$ � crypter. Son encodage num�rique sous forme matricielle est donn� par : $  \begin{pmatrix}
1 \\
2 
\end{pmatrix}$\\\\\\
\textbf{Encryptage} :
$ A \cdot \begin{pmatrix}
1 \\
2 
\end{pmatrix} = \begin{pmatrix}
9 \\
10 
\end{pmatrix}$ qui se traduit par le message $"JK"$\\\\\\
\textbf{D�cryptage} :
$ A^{-1} \cdot \begin{pmatrix}
9 \\
10 
\end{pmatrix} = \begin{pmatrix}
1 \\
2 
\end{pmatrix}$


 \end{document}