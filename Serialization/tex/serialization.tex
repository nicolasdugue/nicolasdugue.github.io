\input{macros_td.tex}


\begin{document}
\feuille {1} 
%\vspace*{1cm}

\exos

 Dans le TD sur la gestion d'un ensemble de {\tt Note}, nous choisissons d'�crire nos repr�sentations d'objets {\tt Note} sous un format texte, chaque ligne repr�sentant une note.
 Il est possible de choisir une approche diff�rente, celle qui permet de stocker des objets \emph{s�rializ�s} dans un fichier. 
 La premi�re m�thode (format texte) a l'avantage de cr�er un fichier lisible par l'homme.
 La seconde m�thode (s�rialization) encode l'objet dans un format illisible. En revanche, elle demande moins de lignes de codes. Par ailleurs, elle est plus g�n�rique : la mani�re de s�rializer un objet est la m�me quelque soit l'objet, il suffit qu'il impl�mente l'interface {\tt Serializable}.
 L'id�e, c'est que c'est Java qui se charge de l'�criture de l'objet et de tous ses champs.\\
 
 
 
 Pour s�rializer un objet, il suffit de :
 \begin{itemize}
	 \item faire en sorte qu'il soit s�rializable ainsi que tous ces champs
	 \item l'�crire dans un fichier en utilisant {\tt FileOutputStream} et {\tt ObjectOutputStream}\\
 \end{itemize}
 
 
 
 Pour retrouver cet objet s�rializ�, il faut :
 \begin{itemize}
	 \item r�cup�rer l'objet sous forme de {\tt Serializable} dans le fichier dans lequel il est stock� en utilisant {\tt FileInputStream} et {\tt ObjectInputStream}
	 \item caster cet objet dans le type qui convient
 \end{itemize}
 
 
 .
 
 \paragraph{Interface Serializable} Faire en sorte que $Note$ et $EnsembleNote$ impl�mentent l'interface $Serializable$.
 \paragraph{Les m�thodes de s�rialization} 
 Ecrire une m�thode qui �crit un objet $Note$ dans un fichier en utilisant : \begin{itemize}
 \item $public$ $FileOutputStream(String$ $name)$
                 $throws$ $FileNotFoundException$ qui ouvre un flux pour �crire dans le fichier d'emplacement $name$\\
 \item $public$ $ObjectOutputStream(OutputStream$ $out)$
                   $throws$ $IOException$ qui permet d'�crire des objets sur le flux $out$\\
 \item $public$ $final$ $void$ $writeObject(Object$ $obj)$
                       $throws$ $IOException$ de la classe {\tt ObjectOutputStream} qui �crit un objet $obj$\\
 \end{itemize}
 
 
 Puis r�cup�rer l'objet $Note$ que l'on vient d'�crire � partir du fichier en vous servant de : 
 \begin{itemize}
 \item $public$ $FileInputStream(String$ $name)$
               $ throws$ $FileNotFoundException$ qui ouvre un flux pour lire dans le fichier d'emplacement $name$ \\
 \item $public$ $ObjectInputStream(InputStream$ $in)$
                  $throws$ $IOException$ qui permet de lire des objets avec le flux $in$ \\
 \item $public$ $final$ $Object$ $readObject()$
                        $throws$ $IOException,$
                               $ClassNotFoundException$ de la classe {\tt ObjectInputStream} qui lit un objet\\
 \end{itemize}
 
 Effectuer le m\^eme processus avec un objet {\tt EnsembleNote}.
 
%\end{figure}
 \end{document}